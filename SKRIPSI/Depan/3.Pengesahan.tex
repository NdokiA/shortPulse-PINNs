%\newpage
%\pagestyle{plain}
%\pagenumbering{roman}
%\setcounter{page}{3}
\chapter*{LEMBAR PENGESAHAN SKRIPSI}
\addcontentsline{toc}{chapter}{LEMBAR PENGESAHAN}

\begin{center}
	\vspace{0.5cm}
	\textbf{OPTIMASI MODEL PHYSICS-INFORMED NEURAL NETWORKS UNTUK MENYELESAIKAN PERSAMAAN SCHRODINGER NON-LINEAR DALAM ANALISIS PERAMBATAN PULSA FIBER OPTIK}\\
	\vspace{0.5cm}
	\textbf{ANDIKO PUTRA PRATAMA KRISDIAWAN}\\
	\textbf{215090301111020}\\
	\vspace{0.5cm}
	Setelah dipertahankan di depan Majelis Penguji \\pada tanggal ... \\ dan dinyatakan memenuhi syarat untuk memperoleh gelar\\ Sarjana  Sains pada bidang Fisika\\
	\vspace{0.5cm}
\end{center}

\begin{center}
\begin{tabular}{c c} % Use 'c' to center columns
	Dosen Pembimbing 1 & Dosen Pembimbing 2\\
	\vspace{1cm} & \vspace{1cm} \\ 
	\underline{\small Prof. Dr.rer.nat. M. Nurhuda} & 
	\underline{\small Prof. Dr. Eng. Agus Naba, S.Si., M.T.} \\
	NIP. 196409101990021001 & NIP. 197208061995121001 \\
\end{tabular}     
\end{center}

\begin{center}
	Mengetahui,\\
	Ketua Departemen Fisika \\Fakultas Matematika dan Ilmu Pengetahuan,\\
	Universitas Brawijaya\\
	\vspace{2cm}
	\underline{Dr. Eng. Masruroh S.Si., M.Si.}\\
	NIP. 197512312002122002
\end{center}

\cleardoublepage
