\chapter*{\textit{OPTIMIZATION OF PHYSICS-INFORMED NEURAL
    NETWORKS MODEL FOR SOLVING NONLINEAR
    SCHRODINGER EQUATION IN OPTICAL FIBER PULSE
    PROPAGATION}}
\addcontentsline{toc}{chapter}{\textit{ABSTRACT}}
\begin{center}
	\vspace{0.5cm}
	\textbf{\textit{ABSTRACT}}
\end{center}
Physics-Informed Neural Networks (PINNs) are novel alternative approach to solve physical systems aside from numerical and data driven methods. This approach combines the flexibility of neural networsk, integrated with physical constraints---such as governing differential equations, initial conditions and boundary conditions---to reduce the extensive data requirements in conventional machine learning models. This research investigates the capability of PINNs in modelling ultrashort pulse in optical fibers using the Nonlinear Schr\"{o}dinger (NLS) Equation. The investigation is carried out by varying the number of collocation samples and random seed value during training process. In addition, new modification of PINNs is proposed by adaptively adding collocation points during training using SMOTE-Adaptive-Sampling PINNs (SAS-PINNs). This approach is intended to help the model learn regions which are difficult to converge. Both models are evaluated via numerical reference using Split-Step Fourier Method (SSFM). Result shows that PINNs accuracy and precision increase exponentially as the number of collocation points increase. Furthermore, SAS-PINNs demonstrates enhanced performance by providing more disperse augmented samples, whereas overly concentrated additions may disrupt overall data distribution.

\noindent \textit{\textbf{Keywords:} Physics Instructed Neural Network, Non-Linear Schrodinger Equation, Optical Fiber}

\cleardoublepage
