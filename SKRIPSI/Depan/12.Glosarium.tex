\chapter*{GLOSARIUM}
\addcontentsline{toc}{chapter}{GLOSARIUM}
\vspace{1cm}
\begin{tabular}{lp{0.75\textwidth}}
\textbf{Istilah} & \textbf{Keterangan} \\
\hline
ADAM & \emph{Adaptive Moment Estimation}, jenis optimasi model jaringan neural yang memanfaatkan akumulasi gradien sebagai momentum. \\
fungsi biaya & Fungsi yang mengukur kesalahan prediksi model terhadap target (dapat disebut sebagai Loss/Fungsi Loss). \\
MSE & \emph{Mean Square Error}, rerata kuadrat selisih antara prediksi model dan nilai target yang digunakan sebagai metrik evaluasi model. \\
LBFGS & \emph{Limited-Memory Broyden-Fletcher-Goldfarb-Shanno}, jenis algoritma optimasi yang mengaproksimasi matriks Hessian (turunan parsial orde-2) invers menggunakan memori terbatas. \\
persamaan NLS & Persamaan Schr\"{o}dinger Non-Linear, model matematis yang menggambarkan evolusi pulsa dalam media nonlinear. \\
PINNs & \emph{Physics-Informed Neural Networks}, jaringan neural yang mengintegrasikan syarat fisis dalam proses pembelajaran melalui penalti residual dari persamaan diferensial. \\
\emph{random seed} & Nilai awal generator bilangan acak untuk memastikan elemen acak dapat direproduksi.\\
SAS-PINNs & \emph{SMOTE-Adaptive-Sampling PINNs}, modifikasi model PINNs dengan menambahkan titik kolokasi pada daerah yang sulit dipelajari model menggunakan teknik SMOTE. \\
\end{tabular}

\begin{tabular}{lp{0.75\textwidth}}
\textbf{Istilah} & \textbf{Keterangan} \\
\hline
SMOTE & \emph{Synthetic Minority Oversampling Technique}, teknik augmentasi data untuk menyeimbangkan kelompok data minoritas dan mayoritas dengan menghasilkan sampel sintetis. \\
SSFM & \emph{Split-Step Fourier Method}, metode numerik berbasis pseudo-spektral untuk menyelesaikan persamaan diferensial dengan memisahkan bagian linear dan nonlinear menggunakan transformasi Fourier. \\
titik kolokasi & Titik sampel dalam domain yang diberikan sebagai input ke model PINNs untuk menghitung residual persamaan diferensial. \\
\emph{ultrashort pulse} & Jenis pulsa dengan durasi waktu yang sangat pendek, mencapai kurang dari 1 ps.\\
\end{tabular}
\cleardoublepage
