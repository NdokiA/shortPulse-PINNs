\chapter*{OPTIMASI MODEL PHYSICS-INFORMED NEURAL NETWORKS UNTUK MENYELESAIKAN PERSAMAAN SCHRODINGER NON-LINEAR DALAM ANALISIS PERAMBATAN PULSA FIBER OPTIK}

\addcontentsline{toc}{chapter}{ABSTRAK}
\begin{center}
	\vspace{0.5cm}
	\textbf{ABSTRAK}
\end{center}
\emph{Physics-Informed Neural Networks} (PINNs) merupakan sebuah metode alternatif dalam menyelesaikan sistem fisis selain pendekatan numerik dan pendekatan berbasis data. Metode ini mengunggulkan fleksibilitas jaringan neural yang diintegrasikan dengan batasan fisis---seperti persamaan diferensial, syarat awal, dan syarat batas---untuk mengurangi kebutuhan data yang besar pada model akal imitasi pada umumnya. Penelitian ini berupaya menginvestigasi kemampuan PINNs dalam memodelkan perambatan pulsa \emph{ultrashort} serat optik melalui persamaan Schr\"{o}dinger Nonlinear (NLS). Hal ini dilakukan dengan memvariasikan jumlah sampel kolokasi dan nilai \emph{random seed} selama proses pembelajaran model. Modifikasi  model dasar PINNs turut diusulkan melalui penambahan titik kolokasi secara adaptif selama pembelajaran melalui \emph{SMOTE-Adaptive-Sampling PINNs} (SAS-PINNs). Sistem ini dirancang untuk membantu model dalam mempelajari zona yang kesulitan mencapai konvergensi. Kedua jenis model dievaluasi berdasarkan nilai referensi melalui metode numerik \emph{Split-Step Fourier Method} (SSFM). Hasil evaluasi menunjukkan bahwa akurasi dan presisi model PINNs meningkat secara eksponensial dengan jumlah titik kolokasi. Di lain sisi, modifikasi model SAS-PINNs ditunjukkan mampu meningkatkan kinerja model PINNs dengan penambahan titik yang lebih tersebar, sementara penambahan secara terfokus berpotensi mengganggu distribusi data keseluruhan.

\noindent\textbf{Keyword:} Physics-Instructed Neural Network, Persamaan Schrodinger Non-Linear, Serat Optik
\cleardoublepage

