%kata pengantar
	\chapter*{KATA PENGANTAR}
	\addcontentsline{toc}{chapter}{KATA PENGANTAR}
	\vspace{1cm}
	
Puji syukur penulis panjatkan ke hadirat Allah S.W.T. karena atas rahmat dan karunia-Nya, penulis dapat menyelesaikan skripsi berjudul "OPTIMASI MODEL PHYSICS-INFORMED NEURAL
NETWORK UNTUK MENYELESAIKAN PERSAMAAN
SCHRODINGER NON-LINEAR DALAM ANALISIS
PERAMBATAN PULSA FIBER OPTIK" ini sebagai salah satu syarat untuk memperoleh gelar Sarjana pada Program Studi Fisika, Fakultas Matematika dan Ilmu Pengetahuan Alam, Universitas Brawijaya. 

Penulis menyadari bahwa skripsi ini tidak dapat terselesaikan tanpa bantuan dan dukungan dari berbagai pihak. Oleh karena itu, penulis menyampaikan rasa terima kasihnya yang mendalam kepada:

\begin{enumerate}[nosep]
    \item Orang tua dan keluarga penulis yang telah senantiasa memberikan rasa kepercayaan, dukungan, dan motivasi yang tak terhingga jumlahnya selama masa kuliah dan proses kepenulisan;
    \item Dr. Eng. Masruroh S.Si., M.Si. selaku Kepala Departemen Fisika Universitas Brawijaya;
    \item Sri Herwiningsih, S.Si., M.App.Sc., Ph.D selaku Ketua Program Studi Fisika Universitas Brawijaya;
    \item Prof. Dr.rer.nat. M. Nurhuda selaku Dosen Pembimbing I yang secara konsisten telah memberikan motivasi, bimbingan, arahan, dan masukan sehingga penelitian ini dapat diselesaikan dengan baik;
    \item Prof. Dr. Eng. Agus Naba S.Si., M.T. selaku Dosen Pembimbing II yang secara konsisten telah memberikan motivasi, bimbingan, arahan, dan masukan sehingga penelitian ini dapat diselesaikan dengan baik;
    \newpage
    \item Universitas Brawijaya yang telah mengizinkan penulis mempergunakan fasilitas AI Center selama masa penelitian;
    \item Rekan-rekan dari jurusan fisika yang telah menjadi bagian dari perjalanan akademik selama masa studi dan penyusunan skripsi;
    \item Semua pihak yang tidak dapat disebutkan satu per satu namun telah memberikan kontribusi, baik secara langsung maupun tidak langsung, dalam penyusunan laporan ini.
\end{enumerate}

Penulis memiliki ketertarikan khusus terhadap pengembangan metode numerik modern, salah satunya adalah Physics-Informed Neural Networks (PINNs), yang menjadi fokus utama dalam penelitian ini.  Maka dari itu, melalui penelitian ini, penulis berharap hasil yang diperoleh dapat memberikan kontribusi yang berarti bagi pengembangan ilmu pengetahuan, khususnya di bidang fisika. Semoga penelitian ini dapat menjadi sumber inspirasi dan acuan bagi peneliti selanjutnya yang tertarik untuk mengeksplorasi topik serupa.

Selama proses kepenulisan penulis telah menemui berbagai tantangan dengan solusi yang tidak terduga. Dengan demikian, sebagai penutup, penulis menyampaikan rasa terima kasih yang tulus atas segala bentuk dukungan dan bantuan yang telah diberikan selama proses penyusunan laporan ini. Semoga segala kebaikan dibalas berlipat ganda oleh Allah SWT.

\begin{flushright}
	Malang, ...\\
	\vspace{1.5cm}
	Andiko Putra Pratama Krisdiawan
\end{flushright}
\cleardoublepage
