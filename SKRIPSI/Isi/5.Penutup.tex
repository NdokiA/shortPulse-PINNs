\addtocontents{toc}{\protect\addvspace{5pt}}
\chapter{PENUTUP}

\section{Kesimpulan}

Berdasarkan hasil penelitian yang dilakukan, dapat disimpulkan bahwa: 

\begin{enumerate}

    \item \emph{Physics-Informed Neural Networks} (PINNs) mampu menyelesaikan persamaan Schr\"{o}dinger Nonlinear (NLS) dalam kasus pulsa \emph{ultrashort}. Metode ini memerlukan normalisasi persamaan untuk menjaga stabilitas numerik. Model menggunakan empat lapisan tersembunyi dengan masing-masing 128 neuron, fungsi aktivasi $\tanh()$, dan strategi optimasi ADAM-LBFGS. Model dites pada variasi jumlah titik kolokasi dan variasi angka \emph{random seed}

    \item Evaluasi Pendekatan Vanilla-PINNs terhadap referensi data \emph{Split-Step Fourier Method} (SSFM) menunjukkan kemampuan prediksi model yang meningkat secara eksponensial dengan jumlah titik kolokasi yang digunakan. 

    \item SAS-PINNs (SMOTE Adaptive Sampling) berhasil meningkatkan akurasi dan konsistensi prediksi dengan mengonfigurasi Nilai Ambang dan Rasio \emph{Oversampling}. Namun, konsentrasi penambahan titik yang terlalu terfokus (niai ambang kecil) berpotensi mengganggu generalisasi model.
    
\end{enumerate}

\newpage
\section{Saran}
Disarankan untuk mengeksplorasi teknik \emph{SMOTE-Adaptive Sampling} (SAS) dengan beberapa modifikasi pelatihan adaptif seperti pendekatan dekomposisi domain maupun penerapan \emph{causal training}.  Kombinasi ini diharapkan meningkatkan performa PINNs dengan mempertimbangkan urutan temporal data.